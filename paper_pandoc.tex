\hypertarget{header}{}
\hypertarget{excitation-transfer-theory}{%
\section{Excitation-Transfer Theory}\label{excitation-transfer-theory}}

\hypertarget{sean-f.-odonovan}{%
\paragraph{Sean F. O'Donovan}\label{sean-f.-odonovan}}

\hypertarget{introduction}{%
\section{Introduction}\label{introduction}}

Excitation-transfer theory is an extension of Schachter's two-factor
theory of emotion and Hull's drive theory. It maintains that emotion
consists of a physical high energy state and a cognitive label. It
suggests that emotional magnitude can be manipulated by a
non-emotion-specific high energy physical state. In particular, it
suggests that this physical excitation can be produced by communication,
physical circumstances (like exercise or other adrenaline inducing
activities), or other emotion. Then, that excitation can later be
mistakenly attributed to a felt emotion, intensifying that emotion. For
example, if someone thinks they've stepped on a snake, and is then shown
that the snake isn't real, they may become quite angry or very amused.
The excitation from the fear transfers to the anger or amusement.

This phenomenon and theory are interesting because they influence the
flow of emotion; communication has the potential to influence and be
influenced greatly by that emotion. Ultimately, to assess whether or not
this theory is useful, we must examine its claims, including that all
emotion is driven by a nonspecific physical excitation. To do this, we
will explain the origins of excitation-transfer theory, discuss its
evolution, and assess its validity in light of a changing landscape of
literature.

\hypertarget{overview-of-excitation-transfer-theory}{%
\section{Overview of Excitation-Transfer
Theory}\label{overview-of-excitation-transfer-theory}}

Excitation-transfer theory tries to explain a relatively frequent
phenomenon. The most basic example of it is when someone is in a high
energy state and is insulted. They most likely will become more
aggressive than they normally would have if the insulted subject was not
in a high energy state. This is the phenomenon demonstrated in {Zillmann
and Bryant (1974)} (Going forward, this high energy state will be
referred to as a state of arousal or excitation).

Excitation-transfer theory was defined in the 1970s by a series of
studies by Dolf Zillmann, describing a theory that both includes and
expands beyond the demonstrable effect in {Zillmann and Bryant (1974)}.
Its claims vary slightly over time, especially because some of the
theories of emotion it is built on, namely Schachter's two-factor theory
of emotion, have issues with replication. There's an incredible amount
of overlap in Hull's drive theory, Schachter's misattribution theory,
and Zillmann's excitation-transfer theory {(Bryant and Miron 2003)}.

Schachter, Hull, and Zillmann all express that emotion has two parts: an
excitation or arousal, and a cognitive label {(Bryant and Miron 2003)}.
Arousal does not disperse quickly despite the ability to switch
cognitive label near instantaneously. Zillmann's excitation-transfer
theory suggests that when there is an initial excitation or arousal
which puts someone into a higher energy state, and then there is another
stimuli which excites that person (not necessarily much at all), the
emotion (cognitive label) that the person applies to that stimuli is
magnified by the residual excitation from the first stimuli. As {Bryant
and Miron (2003)} explains, ``the nonspecific excitation \ldots{}
produced by subsequent stimuli''piggybacks'' prior residual excitation''
(p.~48).

The theory started out with a simple focus on misattribution of
excitation. {Zillmann (1971)} explains that the magnitude of an emotion
can be influenced by the ``level of excitation present at the time,''
which can be ``transferred from a prior to a subsequent state''
(p.~422). This excitation can come from sources including aggressive
communication {(Zillmann 1971)} or exercise {(Zillmann, Katcher, and
Milavsky 1972)}. In theory, it can be transferred to any emotion, as
emotions here are considered cognitive labels for excitation under the
ideas in two-factor theory {(Zillmann 1971, 421)}. Studied examples of
transfer include anger as measured by behavior {(Zillmann 1971)}, humor
appreciation {(Cantor, Bryant, and Zillmann 1974)}, sexual arousal
{(Cantor, Zillmann, and Bryant 1975)}, and others. There has been the
most study on the transfer of excitation with regard to aggression and
anger.

\hypertarget{evolution}{%
\section{Evolution}\label{evolution}}

As research on this phenomenon progressed, so did the theory of emotion.
There was also a recognition of the limits of both two-factor theory and
excitation-transfer theory that led to some refinement. In particular
{Cantor, Zillmann, and Bryant (1975)} recognized a time based component.
Excitation is not recognized as an increase in magnitude of emotion when
the subject perceives the initial stimuli is causing the excitation;
instead there is a phase between the end of the excitation and the end
of the subject's perception of the excitation when transfer is possible
{(Cantor, Zillmann, and Bryant 1975)}.

This limit actually breaks with Schachter's two-factor theory.
Schachter's model expects a subject to attribute emotion cognitively
based on external factors and environment, but in this case, there is a
sense that previous cognition and a person's sense of what influences
them play a role {(Bryant and Miron 2003, 40)}.

In response, {Tannenbaum and Zillmann (1975)} suggests a new three
factor theory of emotion that builds on but refines Schachter's
two-factor theory. Excitation-transfer theory is not changed much, as
this new theory is more or less built around it as described in {Cantor,
Zillmann, and Bryant (1975)}.

\hypertarget{critical-assessment-of-excitation-transfer-theory}{%
\section{Critical Assessment of Excitation-Transfer
Theory}\label{critical-assessment-of-excitation-transfer-theory}}

This theory has changed over its lifetime, but there are many studies
establishing that the phenomenon it studies exists. The biggest problem
with it is its foundation in two-factor theory, because there have been
studies which fail to replicate its foundational experiments; a summary
explains that ``Schachter's (1964a, b) theory is not well supported by
the research, but the available evidence has not necessarily disproven
the theory either'' {(Cotton 1981, 1)}. Because of this failure to
reproduce and relative lack of evidence, it's reasonable to reassess the
two-factor theory's assumption that all emotions have the same
nonspecific arousal. This is a claim that demands thorough examination
and evidence before it should be taken as fact. As {Cotton (1981)}
explains, ``Other theorists do not treat emotion this way; they believe
that different emotions have different physiological concomitants, and
they emphasize cognitive processes to a lesser degree'' (p.~2).

Beyond the criticisms of two-factor theory which initially underpinned
excitation-transfer theory, the main question to ask is whether or not
the theory adequately explains the processes and cause behind the
phenomenon it describes. The theory seems to have enough study and
refinement that it can accurately predict outcomes of excitation
transfer from physical excitation to behavior and emotions, but it's
hard to tell if the internal model for why that transfer happens is
true.

\hypertarget{conclusion}{%
\section{Conclusion}\label{conclusion}}

This accuracy in prediction is perhaps the most important part of the
theory; it allows for the use of excitation-transfer in mass media
{(Bryant and Miron 2003; Tannenbaum and Zillmann 1975)}, communication
{(Zillmann and Bryant 1974)}, and daily life {(Tannenbaum and Zillmann
1975)}. The theory is useful regardless of whether or not the reasoning
behind it is sound.

\begin{table}[]{@{}ll@{}}
\toprule()
HI & This is a table \\
\midrule()
\endhead
text & yup \\
\bottomrule()
\end{longtable}



\end{CSLReferences}
