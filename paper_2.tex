% Options for packages loaded elsewhere
\PassOptionsToPackage{unicode}{hyperref}
\PassOptionsToPackage{hyphens}{url}
%
\documentclass[
  stu]{apa7}
%
\PassOptionsToPackage{style=apa}{biblatex}
%\PassOptionsToPackage{indent=7ex}{parskip}
%\usepackage{indentfirst}
%\usepackage[margin=1in]{geometry}
%% right side page number
%\usepackage{fancyhdr}
%\fancypagestyle{plain}{
%\fancyhf{}
%\renewcommand{\headrulewidth}{0pt}
%\fancyhead[R]{\thepage}
%}
%\pagestyle{fancy}
%\fancyhf{}
%\renewcommand{\headrulewidth}{0pt}
%\fancyhead[R]{\thepage}
%% double spacing
%\usepackage{setspace}
%\doublespacing
\affiliation{Brian Lamb School of Communication}
\course{COM 212: Interpersonal communication}
\professor{Dr. Hayden Barber}
\duedate{March 31, 2022}
%
\title{Excitation Transfer}
\author{Sean F. O'Donovan}
\date{}

\usepackage{amsmath,amssymb}
\usepackage{lmodern}
\usepackage{iftex}
\ifPDFTeX
  \usepackage[T1]{fontenc}
  \usepackage[utf8]{inputenc}
  \usepackage{textcomp} % provide euro and other symbols
\else % if luatex or xetex
  \usepackage{unicode-math}
  \defaultfontfeatures{Scale=MatchLowercase}
  \defaultfontfeatures[\rmfamily]{Ligatures=TeX,Scale=1}
\fi
% Use upquote if available, for straight quotes in verbatim environments
\IfFileExists{upquote.sty}{\usepackage{upquote}}{}
\IfFileExists{microtype.sty}{% use microtype if available
  \usepackage[]{microtype}
  \UseMicrotypeSet[protrusion]{basicmath} % disable protrusion for tt fonts
}{}
\makeatletter
\@ifundefined{KOMAClassName}{% if non-KOMA class
  \IfFileExists{parskip.sty}{%
    \usepackage{parskip}
  }{% else
    \setlength{\parindent}{0pt}
    \setlength{\parskip}{6pt plus 2pt minus 1pt}}
}{% if KOMA class
  \KOMAoptions{parskip=half}}
\makeatother
\usepackage{xcolor}
\IfFileExists{xurl.sty}{\usepackage{xurl}}{} % add URL line breaks if available
\IfFileExists{bookmark.sty}{\usepackage{bookmark}}{\usepackage{hyperref}}
\hypersetup{
  pdftitle={Excitation Transfer},
  pdfauthor={Sean F. O'Donovan},
  hidelinks,
  pdfcreator={LaTeX via pandoc}}
\urlstyle{same} % disable monospaced font for URLs
\setlength{\emergencystretch}{3em} % prevent overfull lines
\providecommand{\tightlist}{%
  \setlength{\itemsep}{0pt}\setlength{\parskip}{0pt}}
\setcounter{secnumdepth}{-\maxdimen} % remove section numbering
\ifLuaTeX
  \usepackage{selnolig}  % disable illegal ligatures
\fi
\usepackage[]{biblatex}
\addbibresource{comPaper2References.bib}

\begin{document}
\maketitle

\hypertarget{introduction}{%
\section{Introduction}\label{introduction}}

\textcite{zillmann08}

\hypertarget{overview-of-theory}{%
\section{Overview of Theory}\label{overview-of-theory}}

\url{https://onlinelibrary.wiley.com/doi/abs/10.1002/9781405186407.wbiece049}

Excitation-transfer theory tries to explain a relatively frequent
phenomenon. The most basic example of it is when someone is in a high
energy state and is insulted. They most likely will become more
aggressive than they normally would have if the insulted subject was not
in a high energy state. This is the phenomenon demonstrated in
\textcite{zilEffectResidualExcitation}.

Going forward, this high energy state will be referred to as a state of
arousal.

Excitation-transfer theory was defined in the 1970s by a series of
studies by Dolf Zillmann, describing a theory that both includes and
expands beyond the demonstrable effect in
\textcite{zilEffectResidualExcitation}. Its claims vary slightly over
time, especially because some of the theories of emotion it is built on,
namely Schachter's two-factor theory of emotion, have issues with
replication. There's an incredible amount of overlap in Hull's drive
theory, Schachter's misattribution theory, and Zillmann's
excitation-transfer theory
\autocite{bryExcitationTransferTheoryThreeFactor}.

Schachter, Hull, and Zillmann all express that emotion has two parts: an
excitation or arousal, and a cognitive label. Arousal does not disperse
quickly despite the ability to switch cognitive label near
instantaneously. Zillmann's excitation-transfer theory suggests that
when there is an initial excitation or arousal which puts someone into a
higher energy state, and then there is another stimuli which excites
that person (not necessarily much at all), the emotion (cognitive label)
that the person applies to that stimuli is magnified by the residual
excitation from the first stimuli. As \textcite{bry}

\hypertarget{critical-assessment-of-theory}{%
\section{Critical Assessment of
Theory}\label{critical-assessment-of-theory}}

\printbibliography[title=Conclusion]

\end{document}
