% Options for packages loaded elsewhere
\PassOptionsToPackage{unicode}{hyperref}
\PassOptionsToPackage{hyphens}{url}
%
\documentclass[
  stu]{apa7}
%
\PassOptionsToPackage{style=apa}{biblatex}
%\PassOptionsToPackage{indent=7ex}{parskip}
%\usepackage{indentfirst}
%\usepackage[margin=1in]{geometry}
%% right side page number
%\usepackage{fancyhdr}
%\fancypagestyle{plain}{
%\fancyhf{}
%\renewcommand{\headrulewidth}{0pt}
%\fancyhead[R]{\thepage}
%}
%\pagestyle{fancy}
%\fancyhf{}
%\renewcommand{\headrulewidth}{0pt}
%\fancyhead[R]{\thepage}
%% double spacing
%\usepackage{setspace}
%\doublespacing
\affiliation{Brian Lamb School of Communication}
\course{COM 212: Interpersonal communication}
\professor{Hayden Barber}
\duedate{April 29, 2022}
%
\title{Excitation-Transfer Theory}
\author{Sean F. O'Donovan}
\date{}

\usepackage{amsmath,amssymb}
\usepackage{lmodern}
\usepackage{iftex}
\ifPDFTeX
  \usepackage[T1]{fontenc}
  \usepackage[utf8]{inputenc}
  \usepackage{textcomp} % provide euro and other symbols
\else % if luatex or xetex
  \usepackage{unicode-math}
  \defaultfontfeatures{Scale=MatchLowercase}
  \defaultfontfeatures[\rmfamily]{Ligatures=TeX,Scale=1}
\fi
% Use upquote if available, for straight quotes in verbatim environments
\IfFileExists{upquote.sty}{\usepackage{upquote}}{}
\IfFileExists{microtype.sty}{% use microtype if available
  \usepackage[]{microtype}
  \UseMicrotypeSet[protrusion]{basicmath} % disable protrusion for tt fonts
}{}
\makeatletter
\@ifundefined{KOMAClassName}{% if non-KOMA class
  \IfFileExists{parskip.sty}{%
    \usepackage{parskip}
  }{% else
    \setlength{\parindent}{0pt}
    \setlength{\parskip}{6pt plus 2pt minus 1pt}}
}{% if KOMA class
  \KOMAoptions{parskip=half}}
\makeatother
\usepackage{xcolor}
\IfFileExists{xurl.sty}{\usepackage{xurl}}{} % add URL line breaks if available
\IfFileExists{bookmark.sty}{\usepackage{bookmark}}{\usepackage{hyperref}}
\hypersetup{
  pdftitle={Excitation-Transfer Theory},
  pdfauthor={Sean F. O'Donovan},
  hidelinks,
  pdfcreator={LaTeX via pandoc}}
\urlstyle{same} % disable monospaced font for URLs
\setlength{\emergencystretch}{3em} % prevent overfull lines
\providecommand{\tightlist}{%
  \setlength{\itemsep}{0pt}\setlength{\parskip}{0pt}}
\setcounter{secnumdepth}{-\maxdimen} % remove section numbering
\ifLuaTeX
  \usepackage{selnolig}  % disable illegal ligatures
\fi
\usepackage[]{biblatex}
\addbibresource{comPaper2References.bib}

\begin{document}
\maketitle

\hypertarget{introduction}{%
\section{Introduction}\label{introduction}}

Excitation-transfer theory is an extension of Schachter's two-factor
theory of emotion and Hull's drive theory. Hull's drive theory explains
that a physical excitation can provide ``drive'' for behaviors which
need it after the stimuli \autocite[
33]{bryExcitationTransferTheoryThreeFactor}. Schachter's two-factor
theory of emotion claims that emotion is made up of a person's
perception of non-emotion-specific internal physical excitation (a high
energy state) and their attribution of a cognitive label (fearful,
amused) based on perception of external, environmental cues \autocite[
33]{bryExcitationTransferTheoryThreeFactor}.

Excitation-transfer theory maintains that emotion consists of a physical
high energy state and a cognitive label. It suggests that emotional
magnitude can be manipulated by a non-emotion-specific high energy
physical state. In particular, it suggests that this physical excitation
can be produced by communication, physical circumstances (like exercise
or other adrenaline inducing activities), or other emotion. Then, that
excitation can later be mistakenly attributed to a felt emotion,
intensifying that emotion. For example, if someone thinks they've
stepped on a snake, and is then shown that the snake isn't real, they
may become quite angry or very amused. The excitation from the fear
transfers to the anger or amusement.

This phenomenon and theory are interesting because they influence the
flow of emotion; communication has the potential to influence and be
influenced greatly by that emotion. Ultimately, to assess whether or not
this theory is useful, we must examine its claims, including that all
emotion is driven by a nonspecific physical excitation. To do this, we
will explain the origins of excitation-transfer theory, discuss its
evolution, and assess its validity in light of a changing landscape of
literature.

\hypertarget{overview-of-excitation-transfer-theory}{%
\section{Overview of Excitation-Transfer
Theory}\label{overview-of-excitation-transfer-theory}}

Excitation-transfer theory was defined in the 1970s by a series of
studies by Dolf Zillmann, describing a theory that both includes and
expands beyond the demonstrable effect in
\textcite{zilEffectResidualExcitation}. Its claims vary slightly over
time, especially because some of the theories of emotion it is built on,
namely Schachter's two-factor theory of emotion, have issues with
replication. There's an incredible amount of overlap in Hull's drive
theory, Schachter's misattribution theory, and Zillmann's
excitation-transfer theory
\autocite{bryExcitationTransferTheoryThreeFactor}.

Schachter, Hull, and Zillmann all express that emotion has two parts: an
excitation or arousal, and a cognitive label
\autocite{bryExcitationTransferTheoryThreeFactor}. The excitation is a
physical state, and does not disperse quickly despite the ability to
switch cognitive label near instantaneously. Going forward, this
physical high energy state will be referred to as excitation or arousal.

Zillmann's excitation-transfer theory suggests that when there is an
initial excitation or arousal which puts someone into a higher energy
state, and then there is another stimuli which excites that person (not
necessarily much at all), the emotion (cognitive label) that the person
applies to that new stimuli is magnified by the residual excitation from
the first stimuli. As \textcite{bryExcitationTransferTheoryThreeFactor}
explains, ``the nonspecific excitation \ldots{} produced by subsequent
stimuli `piggybacks' prior residual excitation'' (p.~48). This is a
relatively frequent pattern. One basic example of it is when someone is
in a high energy state and is insulted. They most likely will become
more aggressive than they normally would have if the insulted subject
was not in a high energy state. This is the phenomenon demonstrated in
\textcite{zilEffectResidualExcitation}.

The theory started out with a simple focus on misattribution of
excitation. \textcite{zilExcitationTransferCommunicationmediated}
explains that the magnitude of an emotion can be influenced by the
``level of excitation present at the time,'' which can be ``transferred
from a prior to a subsequent state'' (p.~422). This excitation can come
from varied sources including aggressive communication
\autocite{zilExcitationTransferCommunicationmediated} or exercise
\autocite{zilExcitationTransferPhysical}. In theory, it can be
transferred to any emotion, as emotions here are considered cognitive
labels for excitation under the ideas in two-factor theory \autocite[
421]{zilExcitationTransferCommunicationmediated}. Studied examples of
transfer include anger as measured by behavior
\autocite{zilExcitationTransferCommunicationmediated}, amusement
\autocite{canEnhancementHumorAppreciation}, sexual arousal
\autocite{canEnhancementExperiencedSexual}, and others. There has been
the most study on the transfer of excitation with regard to aggression
and anger.

\hypertarget{evolution}{%
\section{Evolution}\label{evolution}}

As research on this phenomenon progressed, so did the theory of emotion.
There was also a recognition of the limits of both two-factor theory and
excitation-transfer theory that led to some refinement. In particular
\textcite{canEnhancementExperiencedSexual} recognized a time based
component. Excitation is not recognized as an increase in magnitude of
emotion when the subject perceives the initial stimuli is causing the
excitation; instead there is a phase between the end of the excitation
and the end of the subject's perception of the excitation when transfer
is possible \autocite{canEnhancementExperiencedSexual}.

This limit actually breaks with Schachter's two-factor theory.
Schachter's model expects a subject to attribute emotion cognitively
based on external factors and environment, but in this case, there is a
sense that previous cognition and a person's sense of what influences
them play a role \autocite[ 40]{bryExcitationTransferTheoryThreeFactor}.

In response, \textcite{tanEmotionalArousalFacilitation} suggests a new
three factor theory of emotion that builds on but refines Schachter's
two-factor theory. Excitation-transfer theory is not changed much, as
this new theory is more or less built around it as described in
\textcite{canEnhancementExperiencedSexual}. In particular, it adds the
factor of disposition, describing a person's previous emotional states
and predisposition to emotional states. It keeps Schachter's excitation
factor, and replaces cognitive attribution with a more abstract
appraisal function that allows for previous cognition and a sense of
what influences oneself. One practical application of the new inclusion
of disposition and appraisal in excitation-transfer is the impact of
mass media. \textcite{tanEmotionalArousalFacilitation} explains that
because most people do not believe themselves to be very influenced by
media, the ``misattribution of accruing arousal'' is more likely,
leading to ``transfer effects in his postcommunication behavior''
\autocite[ 187]{tanEmotionalArousalFacilitation}. In other words, media
is likely to influence people's emotional state and behavior, especially
when they do not recognize this themselves.

\hypertarget{critical-assessment-of-excitation-transfer-theory}{%
\section{Critical Assessment of Excitation-Transfer
Theory}\label{critical-assessment-of-excitation-transfer-theory}}

This theory has changed over its lifetime, but there are many studies
establishing that the phenomenon it studies exists. The biggest problem
with it is its foundation in two-factor theory, because there have been
studies which fail to replicate its foundational experiments; a summary
explains that ``Schachter's (1964a, b) theory is not well supported by
the research, but the available evidence has not necessarily disproven
the theory either'' \autocite[ 1]{cotReviewResearchSchachter}. Because
of this failure to reproduce and relative lack of evidence, it's
reasonable to reassess the two-factor theory's assumption that all
emotions have the same nonspecific arousal. This is a claim that demands
thorough examination and evidence before it should be taken as fact. As
\textcite{cotReviewResearchSchachter} explains, ``Other theorists do not
treat emotion this way; they believe that different emotions have
different physiological concomitants, and they emphasize cognitive
processes to a lesser degree'' (p.~2).

Despite excitation-transfer theory's break with two-factor theory, they
still share the nonspecific arousal component of emotion. Many of
Zillmann's early studies focused specifically on aggression
\autocite{zilExcitationTransferCommunicationmediated,zilExcitationTransferPhysical,zilEffectResidualExcitation},
which doesn't necessarily generalize to all emotion. There has since
been study on other emotions, with positive results:
\textcite{canEnhancementHumorAppreciation} found that jokes were
perceived as funnier due to excitation transfer from communications
based on ``excitatory potential'' rather than the positive or negative
tone of the communication. \textcite{canEffectAffectiveState} found an
excitation transfer effect on the appreciation of music. These studies
verify that excitation-transfer theory can effect amusement and music
appreciation, but it's hard to prove that excitation-transfer effects
every emotion. Excitation-transfer theory isn't proven to occur (or
occur equally) in all cases.

Beyond the criticisms of the excitatory component of emotion (in both
two-factor theory and in excitation-transfer), the main question to ask
is whether or not the theory adequately explains the processes and cause
behind the phenomenon it describes. The theory seems to have enough
study and refinement that it can accurately predict outcomes of
excitation transfer from physical excitation to behavior and emotions,
but it's hard to tell if the internal model for why that transfer
happens is true.

\hypertarget{conclusion}{%
\section{Conclusion}\label{conclusion}}

This accuracy in prediction is perhaps the most important part of the
theory; it allows for the use of excitation-transfer in mass media
\autocite{bryExcitationTransferTheoryThreeFactor,tanEmotionalArousalFacilitation},
communication \autocite{zilEffectResidualExcitation}, and daily life
\autocite{tanEmotionalArousalFacilitation}. The theory is useful
regardless of whether or not the reasoning behind it is sound.

\printbibliography

\end{document}
